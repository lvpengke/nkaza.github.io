\documentclass[]{article}
\usepackage{lmodern}
\usepackage{amssymb,amsmath}
\usepackage{ifxetex,ifluatex}
\usepackage{fixltx2e} % provides \textsubscript
\ifnum 0\ifxetex 1\fi\ifluatex 1\fi=0 % if pdftex
  \usepackage[T1]{fontenc}
  \usepackage[utf8]{inputenc}
\else % if luatex or xelatex
  \ifxetex
    \usepackage{mathspec}
  \else
    \usepackage{fontspec}
  \fi
  \defaultfontfeatures{Ligatures=TeX,Scale=MatchLowercase}
\fi
% use upquote if available, for straight quotes in verbatim environments
\IfFileExists{upquote.sty}{\usepackage{upquote}}{}
% use microtype if available
\IfFileExists{microtype.sty}{%
\usepackage{microtype}
\UseMicrotypeSet[protrusion]{basicmath} % disable protrusion for tt fonts
}{}
\usepackage[margin=1in]{geometry}
\usepackage{hyperref}
\hypersetup{unicode=true,
            pdftitle={HW 1: Due Day 1 11:59 PM},
            pdfborder={0 0 0},
            breaklinks=true}
\urlstyle{same}  % don't use monospace font for urls
\usepackage{graphicx,grffile}
\makeatletter
\def\maxwidth{\ifdim\Gin@nat@width>\linewidth\linewidth\else\Gin@nat@width\fi}
\def\maxheight{\ifdim\Gin@nat@height>\textheight\textheight\else\Gin@nat@height\fi}
\makeatother
% Scale images if necessary, so that they will not overflow the page
% margins by default, and it is still possible to overwrite the defaults
% using explicit options in \includegraphics[width, height, ...]{}
\setkeys{Gin}{width=\maxwidth,height=\maxheight,keepaspectratio}
\IfFileExists{parskip.sty}{%
\usepackage{parskip}
}{% else
\setlength{\parindent}{0pt}
\setlength{\parskip}{6pt plus 2pt minus 1pt}
}
\setlength{\emergencystretch}{3em}  % prevent overfull lines
\providecommand{\tightlist}{%
  \setlength{\itemsep}{0pt}\setlength{\parskip}{0pt}}
\setcounter{secnumdepth}{0}
% Redefines (sub)paragraphs to behave more like sections
\ifx\paragraph\undefined\else
\let\oldparagraph\paragraph
\renewcommand{\paragraph}[1]{\oldparagraph{#1}\mbox{}}
\fi
\ifx\subparagraph\undefined\else
\let\oldsubparagraph\subparagraph
\renewcommand{\subparagraph}[1]{\oldsubparagraph{#1}\mbox{}}
\fi

%%% Use protect on footnotes to avoid problems with footnotes in titles
\let\rmarkdownfootnote\footnote%
\def\footnote{\protect\rmarkdownfootnote}

%%% Change title format to be more compact
\usepackage{titling}

% Create subtitle command for use in maketitle
\newcommand{\subtitle}[1]{
  \posttitle{
    \begin{center}\large#1\end{center}
    }
}

\setlength{\droptitle}{-2em}

  \title{HW 1: Due Day 1 11:59 PM}
    \pretitle{\vspace{\droptitle}\centering\huge}
  \posttitle{\par}
    \author{}
    \preauthor{}\postauthor{}
    \date{}
    \predate{}\postdate{}
  

\begin{document}
\maketitle

\section{Exploratory Data Analysis}\label{exploratory-data-analysis}

Use Canvas to submit the HW. Both RMarkdown and the knitted html file is
required

\subsection{Task 1}\label{task-1}

Instead of O\(_3\) as used in the tutorial, perform the analysis for
NO\(_2\) for 2017 data. For this you will need to download the data from
the \href{https://aqs.epa.gov/aqsweb/airdata/download_files.html}{EPA
website} and clean it. You need to tell a story about this pollutant and
its distribution (in various senses). Explore some problematic features
of the data collection and discuss their impacts on your analyses. You
may consider some of the following questions in directing your analyses.

\begin{itemize}
\tightlist
\item
  How many sensors are measuring NO\(_2\)?
\item
  Of what percentage of the year are the NO\(_2\) sensors active? Where
  are the sensors that are not active? Map them.
\item
  How does the geographical distribution of NO\(_2\) AQI differ from
  that of O\(_3\)
\item
  Which cities (CBSA, not counties) are worst affected by NO\(_2\)?
\item
  Is there a correlation between NO\(_2\) and O\(_3\)? Do different
  correlations matter? (i.e.~correlations among AQIs of NO\(_2\) and
  O\(_3\) at site level, vs correlation of days AQI\textgreater{}100 at
  CBSA level, vs correlation of days AQI\textgreater{}100 at CBSA level
  etc.)
\item
  What does the scatterplot look like? What does facetting the scatter
  plot by state tell us (pick 5 or so states)?
\item
  Link this with other data (temperature, population etc.)? Where to
  acquires these datasets? How to link them?
\end{itemize}

These are by no means, exhaustive. Feel free to engage with your
interests and the interesting bits about the datasets.

\subsection{Task 2}\label{task-2}

Download the motor vehicle traffic collisons data from
\href{https://data.cityofnewyork.us/Public-Safety/NYPD-Motor-Vehicle-Collisions/h9gi-nx95}{NYC
Open data portal}. Answer the following questions

\begin{itemize}
\tightlist
\item
  Which locations have high incidences of traffic collisons?
\item
  How are these high traffic collisons locations different at different
  times of the day?
\item
  Visualise the correlation between home values in a block group and
  traffic collisions and tell a story.
\end{itemize}


\end{document}
